The Dataflow visualization applies the idea of parallel coordinates proposed by inselberg1985plane to network traffic. In a similar fashion to the work of Keith Fligg and Genevieve Max, the visualization shows the `flow' of each packet, representing each as a line through the parallel coordinates. This provides an informative view of the whole traffic in the network. It visualises the distribution of various packet attributes while also giving an intuitive insight into relationships and correlation between distinct aspects of the packets.

Lines representing packets are coloured based on their value in some coordinate. They fade out based on how old the packet is, thereby placing focus on newer developments in the network. 

A common criticism of parallel coordinate plots is that they make it hard to interpret data that is uniformly distributed between coordinates or that is very concentrated around few values (applied2004). Our implementation addresses these concerns using animation. Lines representing new packets are randomly pertubed and move subtely. This makes it easier to spot if a colourful line represents one or more packets. The animation allows the user to distinguish between single packets and concentrated groups of packets with the same characteristics.

For example, if multiple requests to a server came from a single MAC address through the same ports and using the same protocol, in a non-animated visualization they would all be represented as the same line. A user would be erroneuosly interpret this as a single request. Adding randomness makes the pack of requests more visible without significantly impacting the accuracy of the data representation.