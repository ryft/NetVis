It displays a graph of the distribution of some packet attribute (eg. Distribution of port ranges across all packets). 
The main graph is on a log scale useful for identifying spikes in the volume of data on some attribute interval. 
Each interval also displays some dots. The area of those dots is an actual representation of the traffic that has that certain attribute property (The dots are also color coded - a red dot is bigger than a blue). 
In this visualisation you can select a range for an attribute and the visualisation will apply a removable data filter that will only allow data in that range to get to the visualisations.

Displays a graph of the distribution of some packet attribute using its normalised value.
It improves upon existing distribution visualizations by offering a dual representation.
There is a logarithmic line graph layer useful for detecting spikes in the data and a layer of circles that indicates the actual distribution (no logarithm applied) through their area. 
Everything is color coded to give an extra indicative of the volume of traffic (red means bigger than blue).
All these ideas have a single purpose: make the analyst understand what is going on. // maybe you can change this into something that makes more sense (combine different ideas into one visualisation...)
Selecting a range of data directly on the visualization applies a filter for that range throughout the application making it a a starting point for normaliser filters. 
