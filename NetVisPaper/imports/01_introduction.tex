Analysing and monitoring network traffic is important in network security. Since the amounts of data involved in this task are extraordinarily large, it can be hard to make effective use of them. Visualizations offer a solution: they give a compact representation of data to a human analyst who can use them to detect patterns in the network \cite{ware2012information}. An effective visualization of network traffic will make it easy to identify outliers while making clear the general patterns of network usage, and it will facilitate detection of intrusions and of malicious activity.

In this paper, we present a set of visualization techniques that work in tandem to improve awareness of activities in ongoing network traffic passing through an analysts systems. They have been implemented in a robust application to provide an analysis framework for 
network activity in an interactive manner; enabling a deeper and more straightforward understanding of the data. Two of the visualizations are based on existing literature (parallel coordinate plots \cite{inselberg1985plane} and spinning cube of potential doom \cite{cube04}), whereas the four remaining visualizations are novel to this paper. These are referred to as: Attribute Distribution, Traffic Volume, Heat Map and Activity Groups.  

Time series data in the form of packet captures are processed in real-time, and simultaneously rendered in multiple connected visualizations. The user can switch between the available representations of the data and dynamically influence both the visualization's layout and the amount and type of the data displayed. The software is designed to make it easy to spot irregular activity and then to investigate it from multiple perspectives.

The framework developed aims to provide an environment in which situational awareness can more effectively be obtained. It is therefore built to be easily extendable to fit the demands of the specific situation. New visualizations will in a natural way complement the existing ones: The underlying data processing engine provides a standard, or \textit{normalised}, data basis which is shared by all displays.

The purpose of the tools presented in this paper is to assist a human observer to make sense of what is happening in a given network. They emphasize exceptions, show comparisons, and try to answer a wide array of question in a concise and simple fashion. We want the user to generate good hypotheses in response to the visualized information. To achieve this, many aids are given to the user in order to avoid cluttering displays with irrelevant data. In particular, we have implemented a powerful filtering system which can be dynamically adjusted in response to changing situations.

The application's main purpose is to monitor the current activity in the network. However, the same architecture can be used to forensically analyse recorded data. The system will simulate the data records as if they represented a live network. This exploits the important dimension of time, and makes it easier to understand recorded activity.
