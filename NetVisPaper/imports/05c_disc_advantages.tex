<<<<<<< HEAD
Existing network visualization solutions are typically single-purpose programs which present given data in one specific way (see for instance the catalogue in  applied2004). The analyst is expected to choose in advance which kind of visualization will provide the most insight, and is then limited to what the specific application is able to show. Existing applications also require different input file formats which makes it complicated to use them simultaneously (for a discussion of these problems see). 

The program presented here takes a single data input stream and simultaneously visualizes it in different presentations. Since these visualizations are built upon a common base, they are able to complement and inform each other. This improves the user's ability to investigate anomalies.

Network visualizations have to process and display a large amount of data. Often, this can cause a visualization to be cluttered and lose its effectiveness. An obvious solution is to provide filters which allow the analyst to focus on phenomena of interest. The filters we provide apply application-wide and can be defined in an intuitive manner from within the visualizations themselves in a ``click-and-zoom'' fashion. This is a significant improvement over existing workflows.

The framework developed here can be used both for real-time monitoring, but also for historical analysis, which makes it applicable for a wide variety of use cases. Furthermore, the application can be used as a learning tool: By observing archived data captures, new users can familiarise themselves with the general patterns of network usage, and can also see how suspicious patterns show up in a visualization.
=======
The complementary nature of the visualisations means that vitually all information a user could want about the network in available somewhere.  It also makes it very easy to spot a point of interest in one visualisation, then follow it through the others to see more information.

The dynamic filtering makes focusing on a point of interest extremely easy, allowing a user to remove any unnecessary or distracting data.  As the filtering is applied retroactively, it allows the user to see past events from different perspectives in order to determine the cause.

As the application works in real-time, the user can see the current state of the network.  Hence, they could see a developing network attack and take preventative measures, e.g. blocking the IP of a client attempting a brute-force SSH attack.

Can be used as a learning tool
>>>>>>> 199c707e839edf9d7e0f9965fc238076bc868a73
