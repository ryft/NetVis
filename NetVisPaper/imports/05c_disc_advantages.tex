The complementary nature of the visualisations means that vitually all information a user could want about the network in available somewhere.  It also makes it very easy to spot a point of interest in one visualisation, then follow it through the others to see more information.

The dynamic filtering makes focusing on a point of interest extremely easy, allowing a user to remove any unnecessary or distracting data.  As the filtering is applied retroactively, it allows the user to see past events from different perspectives in order to determine the cause.

As the application works in real-time, the user can see the current state of the network.  Hence, they could see a developing network attack and take preventative measures, e.g. blocking the IP of a client attempting a brute-force SSH attack.

Can be used as a learning tool