Existing network visualization solutions are typically single-purpose programs which present given data in one specific way (see for instance the catalogue in \cite{marty2009applied}). The analyst is expected to choose in advance which kind of visualization will provide the most insight, and is then limited to what the specific application is able to show. Existing applications also require different input file formats which makes simultaneous use complicated.

The program presented here takes a single data input stream and simultaneously visualizes it in different presentations. Since these visualizations are built upon a common base, they are able to complement and inform each other. This improves the user's ability to investigate anomalies.

Network visualizations have to process and display a large amount of data. Often, this can cause a visualization to become cluttered. An obvious solution is to provide filters which allow the analyst to focus on phenomena of interest. The filters we provide apply application-wide and can be defined in an intuitive manner from within the visualizations themselves in a ``click-and-zoom'' fashion. This is a significant improvement over existing workflows.

The framework developed here can be used both for real-time monitoring and for historical analysis, which makes it applicable to a wide variety of use cases. Furthermore, the application can be used as a learning tool: by observing archived data captures, new users can familiarise themselves with the general patterns of network usage and can see how suspicious patterns show up in a visualization.