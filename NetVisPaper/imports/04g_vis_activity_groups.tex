Acting on the hexagonal grid, this visualization puts to use two important visual factors:
size and proximity.

As the traffic in the network arises the Activity visualization groups together all the machines
sending packets to a one specific other device. As more and more machines communicate with a specific
address, more and more nodes appear around the hexagon representing the that destination.

Color of the communicating nodes represents how much data they send. More commonly used destinations 
will then have more orange and red nodes, and destinations used by a lot of machines will have more
nodes around them. Therefore, both a size of the segment and its color indicate important information.

The placement of the nodes is automatic and it always assures that there is enough space around a center
and that no two nodes overlap.

This visualization helps detecting the machines that perform a server role in the network, as well as
it allows a user to quickly guess what type of service is the device providing. Each node specifies the
most commonly used protocol, thus revealing the possible role of the server node.
