\includegraphics[width=\linewidth]{materials/groups.jpg}
Expanding on the hexagonal grid, this visualization employs two important visual factors:
size and proximity.

As the traffic in the network rises, the visualization groups together all the machines
sending packets to one specific device. As more and more machines communicate with a specific
address, more and more nodes appear around the hexagon representing that destination.

The colour of the communicating nodes represents how much data they send. More heavily used destinations 
will thus have more orange and red nodes around them, and destinations used by a lot of machines will have a greater number of nodes around them.

Node placement is automatic. The procedure ensures that there is enough space around a centre and that no two nodes overlap.

This visualization helps detect the machines that perform a server role in the network.
It also allows a user to quickly guess what type of service the device is providing. Each node specifies the
most commonly used protocol, thus revealing the possible role of the server node.