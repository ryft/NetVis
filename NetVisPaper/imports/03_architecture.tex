\textbf{Add information on CSV, Data, non-existence of analysis of packet content, and add info on data flow through the application.}

Normalisers:
A normaliser is a class responsible with taking a packet and returning a ``normalised'' value of a certain attribute of that packet and the other way around. The Source IP normaliser provides a method that takes a packet and returns a value between 0 and 1 corresponding to its Source IP (0.0.0.0 returns 0, 255.255.255.255 returns 1). It is also capable of taking a value from 0 to 1 and returning a human readable IP address. 
Each normalising class is able to create a temporary filter in the application that will filter its corresponding attribute on a certain range. 
Since the normalising class is in control of its filter it can create a zooming effect based on the value of the filter. 
(If there is a filter on the Source Port normaliser from 0.5 to 1 then port 30000 will be normalised to ~0 instead of ~0.5.)
// You can extend this + make it pretty

Filtering:
\textbf{High-level: What do filters do?}
The application supports two types of filters - filters that the user explicitly defines, and filters applied on-the-fly from within the visualisations.
Both types are applied to all visualisations and information displayed, and can be adjusted at any time without losing prior data.
The user has access to a variety of different filter controls.  First, there is a menu for filtering by transit protocol. By default, all protocols are selected and therefore included. Protolcols are sorted into menus by protocol family, appearing in multiple places where appropriate. Second, there is a control to select the range of ports the application uses, which defaults to the maximum port range.  The source and destination ports can be set separately, enabling a user to view all data entering/exiting a port as desired.  Next, there are IP and MAC address filters.  These work on a blacklist/whitelist system, allowing a user to only view packets to or from a particular set of addresses, or ignore packets going to or from a different set.  This enables a user to, for instance, ignore traffic from sources they know are irrelevant or focus only on an address that is causing concern.
The second type of filter will be dealt with in more detail in sections 4.2 and 4.3.

Modularity: \textbf{[Make this better.]}
The application was designed to be very extensible.  Adding a new visualisation is simply a matter of adding it to the application's list of those available, which will cause it to be included and kept up-to-date as packets come in and are filtered.  The same is true for filters and normalisers.  \textbf{[wrong place:] }Adding a filter would result in its controls automatically being included in the right panel and all packets would then be filtered according to the criteria it specifies.  Adding a normaliser would cause it to be integrated with the Spinning Cube, Dataflow and Attribute Distribution visualisations.  Ths modularity extends even so far as data input.  If a class were written to accept packets from a different source, it would be trivial to switch the application to use this class.