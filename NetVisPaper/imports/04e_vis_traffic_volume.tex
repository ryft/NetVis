The traffic volume visualization is inspired by the `FlowScan'
<<<<<<< HEAD
graph (\url{www.caida.org/tools/utilities/flowscan/}}).
It is realised as a stacked bar chart which displays the volume of data arriving
in each time interval. Each column is segmented into segments with heights proportional to the total
=======
graph\footnote{See \url{www.caida.org/tools/utilities/flowscan/}}.
It's realised as a stacked bar chart which displays the volume of data arriving
in each time interval, with column segment heights proportional to the total
>>>>>>> 199c707e839edf9d7e0f9965fc238076bc868a73
number of packets transmitted for each protocol. This allows users to see at a
glance if a particular protocol is being exploited in the network. In addition,
the column segments are colour-coded and can be cross-referenced with a
protocol key underneath the visualization itself.

To help distinguish different protocols, colours are selected from a colour
palette which provides colours from a qualitative colour scheme.
This visualization provides a control in the right panel which lets the user
adjust the number of time intervals displayed at once.
