The Traffic Volume visualization is inspired by the `FlowScan' graph
(\url{www.caida.org/tools/utilities/flowscan/}). It is realised as a stacked
bar chart which displays the volume of data arriving in each time interval.
Each column is segmented into segments with heights proportional to the total
number of packets transmitted for each protocol.
In addition, the column segments are colour-coded and can be cross-referenced
with a protocol key underneath the visualization itself.

To help distinguish between different protocols, colours are selected from a
colour palette which provides colours from a qualitative colour scheme.
The objective of this visualisation is to allow users to see at a glance if a
particular protocol is being exploited in the network. The increase in both
column height and colour proportion should draw attention to any protocol which
becomes overburdened.

A control panel is provided in the right panel for the purpose of adjusting the
number of time intervals displayed at once on the x-axis. The y-axis scales
automatically to fit the relevant data.
