\documentclass[12pt,a4paper]{article}

\author{James Nicholls / Group 4}
\title{Packet Capture Specification}
\date{\today}

\usepackage{fullpage}
\usepackage{url}
% Define a double-space function
\newcommand{\dbl}{\\[\baselineskip]}

\setlength{\parindent}{0cm}

\begin{document}

    \maketitle

    This document provides a specification and index of the CSV files
    which are in use as test data for the network visualisation
    application NetVis.

    \section{File format specification}
        All capture files are provided to the application as CSV files
        with the following headers;

    \begin{itemize}
    \setlength{\itemsep}{0pt}
        \item \verb!No.          ! Packet number
        \item \verb!Time         ! Time elapsed since first packet (seconds)
        \item \verb!Source IP    ! Source IPv4/6 address
        \item \verb!Source HW    ! Source hardware (MAC) address
        \item \verb!Source Port  ! Source port
        \item \verb!Dest IP      ! Destination IPv4/6 address
        \item \verb!Dest HW      ! Destination hardware (MAC) address
        \item \verb!Dest Port    ! Destination port
        \item \verb!Protocol     ! Communication protocol
        \item \verb!Length       ! Packet length (bytes)
        \item \verb!Info         ! Detected description of packet purpose
    \end{itemize}

    \section{Sources}

    Notable external sources of packet trace (pcap) files.

    \begin{itemize}
        \item \url{https://www.evilfingers.com/} \\
            A community portal for Information Security, who publish
            internet security papers and keep a public archive of
            PCAP samples, among other resources.

        \item \url{http://www.honeynet.org/} \\
            The Honeynet Project is a leading international 501c3
            non-profit security research organization, dedicated to
            investigating the latest attacks and developing open source
            security tools to improve Internet security.
    \end{itemize}

    \pagebreak

    \section{Capture files}
        This section comprises a list of CSV files currently in use in
        application development and testing, as well as a short
        description of each.

        \subsection{eduroam.csv}
            Source: J. Nicholls

            Original filename: \verb!eduroam.pcap!

            Size: 85664 packets - 16.4 MB \dbl
            All traffic seen by an Ubuntu laptop with minimal running
            services, connected to the Eduroam network on the wlan0
            interface. Approximately 85000 packets over 35 minutes.

        \subsection{jre-overflow.csv}
            Source: \url{https://www.evilfingers.com/}

            Original filename:

            \verb!        Sun_jre1.6.0_X_isInstalled.dnsResolve_Function_Overflow_PoC.pcap!

            Size: 65561 packets - 14.7 MB \dbl
            Proof-of-concept packet capture of a denial of service
            attack on JRE 1.6.0 by exploiting the DNS resolution
            function. A local server is flooded with 65000 packets in
            11 minutes.

        \subsection{port-scan.csv}
            Source: J. Happa

            Original filename: \verb!portscan.pcap!

            Size: 1818 packets - 371 kB \dbl
            A port scan of a Windows Vista PC, originating from an
            Ubuntu PC, concluding that only port 80 (http) is open.

        \subsection{remote-execution.csv}
            Source: \url{http://www.honeynet.org/}

            Original filename: \verb!attack-trace.pcap_.gz!

            Size: 348 packets - 53.6 kB\dbl
            Packet trace of a malware attack which distributes a
            payload exploiting the Windows Local Security Authority
            (LSA) Remote Procedure Call (RPC) service of the victim
            host, compromising the IPC\$ share. Once the share is
            exploited, a script is invoked, causing a connection to an
            FTP server named NzmxFtpd and the acquisition of an
            infected executable, \verb!ssms.exe!.

        \subsection{skype.csv}
            Source: J. Nicholls

            Original filename: \verb|skype.pcap|

            Size: 418 packets - 73 kB\dbl
            Packets transferred during the authentication and
            initialisation of a Skype session.\\ Recorded on an Ubuntu
            PC with minimal services running.

        \subsection{ssh-attack.csv}
            Source: \url{http://www.honeynet.org/}

            Original filename: \verb|hp_challenge.pcap|

            Size: 5447 packets - 951.2 kB\dbl
            Packet trace of an intruder gaining access to a server
            using a brute-force attack via SSH, before planting malware
            to download and execute software on the compromised host.

        \subsection{telnet-freebsd-exploit.csv}
            Source: \url{http://www.honeynet.org/}

            Original filename: \verb|fc.pcap|

            Size: 238 packets - 35.2 kB\dbl
            Demonstration of a buffer overflow exploit (CVE-2011-4862)
            that allows arbitrary code execution on a vulnerable
            FreeBSD server via telnet.

        \subsection{ubuntu-update.csv}
            Source: J. Happa

            Original filename: \verb|ubuntu-update.pcap|

            Size: 497 packets - 84.6 kB\dbl
            Packet trace of an Ubuntu PC communicating with a Canonical
            server to check for software updates. No new updates were
            found or downloaded.

        \subsection{nitroba.csv}
            Source: \url{http://digitalcorpora.org/corpora/scenarios/}

            \verb!        nitroba-university-harassment-scenario!

            Original filename: \verb|nitroba.pcap|

            Size: 95175 packets - 17.2 MB\dbl
            Digital forensics excercise scenario involving a large
            packet capture from a shared wireless router, in a case of
            teacher harassment. See the above URL for details of the
            exercise.

\end{document}

