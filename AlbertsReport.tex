\documentclass[12pt,a4paper]{article}
\usepackage{fullpage}
\usepackage{parskip}
\usepackage{url}
% Define a double-space function
\newcommand{\dbl}{\\[\baselineskip]}

\author{Albert ``Errendir'' Siddhartha Slawinski}
\title{Individual Report from the project ``Computer Security Visualisation''}
\date{\today}

\begin{document}
	
	\maketitle

    \section*{Another story ends...}
	As a member of the Clockwork Dragon Team I've been working on the Security Visualisation project for over three months. In this time the five of us became the rock band of developers, and now our hit is getting platinum.
	Three and a half thousands of lines of code might be more than I've ever written for a single cause in my life. There is no world in which I could have done this without the support and concentration of the team, errorless distribution of responsibilities and fully devoted fans.
	Now when we collect our award there is only one thing left to say: ``We could have done better''. But also we could have done less better than other teams and that's always the factor that decides the winner.
	
	Two of the visualisations we presented are authored by me. Both of them are related and both put to use the same idea of hexagonal tesselation of the plane.
    \section*{A road towards implementation}
	First rough implementation of the hexagonal map was commited to github on 16th of April and was far different from the final product. Had I known how I would end up stitching it all together I would take a completely different route.
	That is really what separates a great programmer from a merely good one \textemdash  which I am neither. A great programmer refuses to write a single line of code or pseudocode without having a complete picture of the task. Had we been less of a rock band, jamming in the garage, we would do much better. But that is only clear in the hindsight.
	
	
	
	In the weeks following my first commit I was busy trying to get a hold of OpenGL and trying to bring back the nostalgic memories of the 15 years old me drawing a single triangle with DirectX. I split my program into clearly separated subroutines, but completely ignored any knowledge about object oriented design I had.
	Finally when the first rough version, with the first rough graphics was made I realized that I need to changer something...
    \section*{Design? What is that?}
	It's always good to show your work to somebody who can distance himself/herself from it. Just like a rock band that need this one person that will always say what on their mind, I needed somebody from outside of the group.
	Somebody to asses the riffs I play and tell my why thay don't harmonize. Our team had a great wisdom to split our work in a way that allows to 'mute' all the other instruments and just listen to one. With a basic network-packet logic we could have very easily develop indepentet visualisations.
	That made it really easy for me to separate ``my piece'' and show it to my brother who gave me a very important lesson about design:
	\begin{itemize}
		\item Be clear
		\item Lapidarity starts familiarity
		\item Lead the users to what you want them to see
	\end{itemize}
      

        \section*{Final Product}
                After making some hard design decisions and repackaging the code in the proper Object Oriented way I was close to finilising my work. Both the visualisations were working and after the whole night coding session in the basements of the CS department the tiling algorithm was ready. This meant that the hexagons I was so tidiously putting together could finally be grouped and placed on the dynamicaly scaling map.

        \section*{We're live in 4... 3... 2...}
                Of course the most important part was still ahead of us. The Presentation. The Big Day. I am proud to say that when we came down to the CS department that day, we came down prepared. Suited up, with my very heavy but powerful PC and one hell of the gig to perform. What played the most important role in the process of impressing the judges was concentration. Concentration on the task of fully presenting our apps capabilities. All of us stood by our product and unlike other teams didn't wonder around looking for a shiny object to look at. We explained our app to anyone who approached and we were honest disregarding who we talked to. Also the factor of ``Actually having something to show'' played important role. Lots of other teams were just showing mockup (second game - has anyone seen it?), not-working prototypes (mapping robot) or quickly duct-taped together park-assist devices.

                It still was a shock when we won. We expected to do well, but not winningly well. As we were told later on it's the versatility of our app that really spoke to the judges.

    
	\section*{Problems We Encountered}

        Of course it's quite a shame that so much work was just to make the OpenGL behave. Of course this knowledge will be useful, but there maybe we could have been better off using some already existing display/game engine.

	\section*{Team assesment}
		Let me now give a brief assesment of every single member of my group:
		
                James quickly became and unofficial manager of our band. Scheduling our meeting, writing to the sponsors, assiging tasks was his daily job. He presented great leadership skills and perfect overview of the project. Managing github issues and mile
		
		Dominik was the one who always made sure that there is a decision made on a meeting. He did a great job writting down reports and specifications.
		
		Sergiu turned out to be the dark horse of our team. With great coding skills and brilliant ideas he contributed wonderful functionalities. He also had a really hard job of tying few visualisations together which he did perfectlty
		
		Thomas gave a presentation as wonderful as the ones given by Steve Jobs. It's not too much to say that we owe him our victory. He also played a very important role in developing early stage logic, which allowed other team members to work in a stable and versitile environment.
		
		Albert (that is of course me) presented nicely looking visualisations, and even though his work was always a bit ``on the side'' his code found a place in the program and in the presentation and gave a non neglectable impression on the viewers.
	\section*{Reflections}
		In the ideal team, does anyone learn anything? Truli understanding something means that you can do it by yourself and therefore you don't need a team.
		
		This is not entirely true. A complex project consists of multiple scopes and I think each of us found his scope and learned something about it.

	
	\section*{Crossroads of destiny?}
                What will happen to the project now? What path will any of us choose?


                As it happens to almost every rock band, there comes a time to go solo. Will our separate work be as good as colective?


                No way of telling...
	
\end{document}
