To this day, most visualization approaches provide a segmented view of a dataset. These 
provide different perspectives on the same data. Some of these may favour graph-based 
representations, port-scanning activity characteristics, network traffic patterns, payload 
characteristics or event-log forensics. Conti \cite{Conti} and Marty \cite{marty2009applied} 
provide a detailed discussion on this topic. 

Some examples of available tools include the open source Rumint \cite{rumint} and Wireshark
\cite{wireshark} for traffic forensics. Many tools for analysing network usage patterns
exist \cite{best2010, RADAR, lau2004spinning, liao2010}. Other approaches include geographical-based
representations of malware activity, such as Sony Rootkit Global Spread \cite{Conti}, and on
city-level visability by Yu et al. \cite{yu2010}. The potential wider impact of attacks on network
assets was visualised by Chu et al. \cite{chu2010}. Intrusion-detection event-correlations have also
been visualized by Rasmussen et al. \cite{rasmussen2010} and Yelizarov et al. \cite{yelizarov2009}.

The commercial SecureScope tool \cite{securescope} addresses business impact of attacks by mapping
clusters of potentially malicious network activity to business role or organisational units (such as
\textit{Human Resources}) and geographical location (such as the \textit{New York Office}).
Similarly, the commercial Arcsight tool \cite{arcsight} provides mapping between event alerts,
source IPs and business role. Another software application is the Tenable 3D Tool \cite{Tenable},
which can visualise topology based on vulnerability scans and change the visible features of
machines accordingly. These features encode information about vulnerabilities, missing patches, open
ports, firewalls, intrusion detection system alerts, netflow, etc.

While many tools exist to enable situational awareness, few works  
have discussed how multiple visualization techniques can complement each other and work in tandem 
to deliver an improved situational awareness from network activity data. This is the main problem 
NetVis addresses.
